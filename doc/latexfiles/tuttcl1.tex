%----------------------------------------------------------------------------
% Magic Tcl tutorial number 1
%----------------------------------------------------------------------------

\NeedsTeXFormat{LaTeX2e}[1994/12/01]
\documentclass[letterpaper,twoside,12pt]{article}
\usepackage{epsfig,times}

\setlength{\textwidth}{8.5in}
\addtolength{\textwidth}{-2.0in}
\setlength{\textheight}{11.0in}
\addtolength{\textheight}{-2.0in}
\setlength{\oddsidemargin}{0in}
\setlength{\evensidemargin}{0pt}
\setlength{\topmargin}{-0.5in}
\setlength{\headheight}{0.2in}
\setlength{\headsep}{0.3in}
\setlength{\topskip}{0pt}

\def\hinch{\hspace*{0.5in}}
\def\starti{\begin{center}\begin{tabbing}\hinch\=\hinch\=\hinch\=hinch\hinch\=\kill}
\def\endi{\end{tabbing}\end{center}}
\def\ig{\>}
\def\ih{\>\>}
\def\ii{\>\>\>}
\def\mytitle{Magic Tcl Tutorial \#1: Introduction}

%----------------------------------------------------------------------------

\begin{document}

\makeatletter
\newcommand{\ps@magic}{%
	\renewcommand{\@oddhead}{\mytitle\hfil\today}%
	\renewcommand{\@evenhead}{\today\hfil\mytitle}%
	\renewcommand{\@evenfoot}{\hfil\textrm{--{\thepage}--}\hfil}%
	\renewcommand{\@oddfoot}{\@evenfoot}}
\newcommand{\ps@mplain}{%
	\renewcommand{\@oddhead}{}%
	\renewcommand{\@evenhead}{}%
	\renewcommand{\@evenfoot}{\hfil\textrm{--{\thepage}--}\hfil}%
	\renewcommand{\@oddfoot}{\@evenfoot}}
\makeatother
\pagestyle{magic}
\thispagestyle{mplain}


\begin{center}
  {\bfseries \Large \mytitle} \\
  \vspace*{0.5in}
  {\itshape R. Timothy Edwards} \\
  \vspace*{0.5in}
   Space Department \\
   Johns Hopkins University \\
   Applied Physics Laboratory \\
   Laurel, MD  20723 \\
  \vspace*{0.25in}
  This tutorial corresponds to Tcl-based Magic version 7.2 \\
\end{center}
\vspace*{0.5in}

\section{What is Tcl-based Magic, and Why?}

In Magic version 7.0, Rajit Manohar incorporated a SCHEME interpreter into
the Magic source, noting the limitation of magic to handle definitions
and variables, conditionals, and block structures.  By embedding an
interpreter into the code, the interpreter's functions are made available
on the magic command line, making magic extensible.  The SCHEME interpreter
and various extensions incorporated into loadable scripts are outlined
in the tutorials {\ttfamily tutscm1.ps} through {\ttfamily tutscm4.ps}.

While making Magic considerably more flexible, the embedded SCHEME
interpreter had some notable drawbacks.  The primary one is that the
SCHEME language is syntactically different from Magic's command-line
syntax.  Also, the interpreter is largely disconnected from the code,
and does not affect or extend the graphics or handle results from magic
commands.

Beginning in Magic version 7.2, Magic has been recast into a framework
called {\it ScriptEDA}, in which existing applications become
{\bfseries extensions} of an interpreter rather than having an interpreter
embedded in the application.  The main advantage of extending over
embedding is that the application becomes a module of the interpreter
language, which does not preclude the use of additional, unrelated
modules in the same interpretive environment.  For example, in Tcl-based
Magic, graphics are handled by Tk (the primary graphics package for
use with Tcl), and applications such as IRSIM (the digital simulator)
can be run as if they were an extension of magic itself.  Commands for
Tcl, Tk, IRSIM, BLT, and any other Tcl-based package can be mixed on
the magic command line.

While {\it ScriptEDA} suggests the use of the {\bfseries SWIG} package to give
applications the ability to be compiled as extensions of any interpreter
(Tcl, Python, SCHEME, Perl, etc.), there are specific advantages to
targeting Tcl.  Foremost, the syntax of Tcl is virtually 100\% compatible
with that of Magic.  This is not coincidentally because both Magic and
Tcl were written by John Ousterhout!  Many ideas from the development of
Magic were incorporated into the overall concept and design of the Tcl
interpreter language.

Largely due to the syntactical compatibility, Tcl-based magic is
completely backwardly-compatible with the non-interpreter version of
magic.  Either can be selected at compile-time, in addition to compiling
with embedded SCHEME, which has been retained in full as an option.
A few minor issues, such as the appearance of the cursor when magic is
used without a Tk console window, are addressed below.

Magic extensions under Tcl are considerable, and explanations of the
features have been split into several tutorial files, as listed in
Table~\ref{tutorials}.

\begin{table}[ht]
   \begin{center}
      \begin{tabular}{|l|} \hline
	Magic Tcl Tutorial \#1: Introduction \\
	Magic Tcl Tutorial \#2: The GUI Wrapper \\
	Magic Tcl Tutorial \#3: Extraction and Netlisting \\
	Magic Tcl Tutorial \#4: Simulation with IRSIM \\
	Magic Tcl Tutorial \#5: Writing Tcl Scripts for Magic \\ \hline
      \end{tabular}
   \end{center}
   \caption{The Magic Tcl tutorials and other documentation.}
   \label{tutorials}
\end{table}

\section{Features of Tcl-based Magic}

In summary, the features of Tcl-based Magic (corresponding to Magic version
7.2, revision 31) are as follows:

\begin{enumerate}
   \item The command name {\bfseries magic} itself is a script, not a compiled
	executable.  The script launches Tcl and gives it the name of a Tcl
	script to evaluate ({\ttfamily magic.tcl}).  The Tcl script loads
	Magic as an extension of Tcl and performs all other operations
	necessary to start up the application.
   \item Command-line arguments passed to Magic have been changed.  Some older,
	unused arguments have been removed.  Several arguments have been
	added, as follows:
	\begin{enumerate}
	   \item {\ttfamily -noconsole}  Normally, under Tcl/Tk, Magic starts
		by launching a Tk-based console window ({\ttfamily tkcon.tcl})
		which is the window that accepts magic commands.  Previous
		versions of Magic accepted commands from the calling terminal.
		The former style of running commands from the calling terminal
		can be selected by choosing this argument at runtime.  Note,
		however, that due to fundamental differences in the underlying
		input routines between Tcl and magic, the terminal-based
		command entry does not exactly match the original behavior
		of magic.  In particular, the background DRC function does
		not change the prompt.  In addition, ``Xterm'' consoles
		must select option ``Allow SendEvents'' for keystrokes to
		be echoed from the layout window into the terminal.  For
		security reasons, the application cannot change this option
		on the terminal.
	   \item {\ttfamily -wrapper}  To enforce backward-compatibility,
		magic appears as it would without the Tcl interpreter 
		(apart from the text entry console) when launched with the
		same arguments.  However, most users will want to make use
		of the extensions and convenience functions provided by
		the ``wrapper'' GUI.  These functions are detailed in a
		separate tutorial (see Tutorial~\#2).
	\end{enumerate}
   \item Magic-related programs ``ext2spice'' and ``ext2sim'' have been
	recast as magic commands instead of standalone executables.  This
	makes a good deal of sense, as both programs make heavy use of
	the magic internal database.  Most values required to produce
	netlists were passed through the intermediary file format {\ttfamily
	.ext}.  However, not all necessary values were included in the
	{\ttfamily .ext} file format specification, and some of these
	missing values were hard-coded into the programs where they have
	since become mismatched to the magic database.  This has been
	corrected so that all netlist output corresponds to the technology
	file used by a layout.  

	Both commands ``ext2spice'' and ``ext2sim'' allow all of the
	command-line arguments previously accepted by the standalone
	programs.  Some of the more common functions, however, have
	been duplicated as command options in the usual format of
	magic commands.  For instance, one may use the magic command:
	\starti
	   \ii {\bfseries ext2sim help}
	\endi
	to get more information on {\ttfamily ext2sim} command-line
	options.
   \item All Tcl interpreter procedures may be freely mixed with magic
	commands on the command line.  For instance, a user can use
	Tcl to compute arithmetic expressions on the command line:
	\starti
	   \ii {\bfseries copy e [expr\{276 * 5 + 4\}]}
	\endi
   \item A number of magic commands pass values back to the interpreter.
	For instance, the command
	\starti
	   \ii {\bfseries box values}
	\endi
	returns the lower left- and upper right-hand coordinates of the
	cursor box in magic internal units.  This return value can be
	incorporated into an expression, such as the one below which
	moves the box to the left by the value of its own width:
	\starti
	   \ii {\bfseries set bbox [box values]} \\
	   \ii {\bfseries set bwidth \[expr \{[lindex \$bbox 2] - [lindex
			\$bbox 0]\}]} \\
	   \ii {\bfseries move e \$bwidth}
	\endi
   \item Magic prompts for text have been recast as Tk dialog boxes.
	For instance, the command {\bfseries writeall} will pop up a
	dialog box with five buttons, one for each of the available
	choices (write, flush, skip, abort, and auto).
   \item IRSIM is no longer available as a ``mode'' reached by switching
	tools by command or the space-bar macro.  Instead, IRSIM (version
	9.6) can be compiled as a Tcl extension in the same manner as
	Magic, at compile time.  When this is done, IRSIM is invoked
	simply by typing 
	\starti
	   \ii {\bfseries irsim}
	\endi
	on the Magic command line.  IRSIM is loaded as a Tcl package,
	and IRSIM and Magic commands may be freely mixed on the Magic
	command line:
	\starti
	   \ii {\bfseries assert [getnode]}
	\endi
	IRSIM does not require the name of a file to start.  Without
	arguments, it will assume the currently loaded cell is to be
	simulated, and it will generate the {\ttfamily .sim} file if
	it does not already exist.

   \item In addition to IRSIM, programs {\bf xcircuit} and {\bf
	netgen} can be compiled as Tcl extensions and provide
	cabability for schematic capture and layout-vs.-schematic,
	respectively. 

   \item Tcl-based Magic makes use of the Tcl variable space.  For
	instance, it keeps track of the installation directory
	through the variable {\ttfamily CAD\_HOME}, which mirrors
	the value of the shell environment variable of the same
	name.  In addition, it makes use of variables {\ttfamily
	VDD} and {\tt GND} in the technology file to aid in the
	extraction of substrate- and well-connected nodes on
	transistors.

   \item Magic input files, such as the {\ttfamily .magic} startup
	file, are sourced as Tcl scripts, and so may themselves
	contain a mixture of Tcl and magic commands.

\end{enumerate}

\section{Compilation and Installation}

   Magic is selected for compilation as a Tcl extention when running
   the 
   \starti
      \ii {\bfseries make config}
   \endi
   script.  The first question asks whether magic should be compiled
   as a Tcl extension, with the SCHEME interpreter embedded, or as
   the original version with no interpreter.  Subsequent to choosing
   Tcl as the interpreter and answering the remaining configuration
   questions, Tcl should be compiled and installed using the commands
   \starti
      \ii {\bfseries make tcl}
   \endi
   and
   \starti
      \ii {\bfseries make install-tcl}
   \endi
   Note that if magic is to be compiled under different interpreters,
   it is necessary to perform
   \starti
      \ii {\bfseries make clean}
   \endi
   between the compilations, so that all references to the last
   compiled version are deleted.

\section{Dual-Source Input and Backward Compatibility}

From its inception, Magic has used an unsual but very effective interface
in which commands may be passed to the program from two different sources,
the layout window, and the calling terminal.  Keystrokes in the layout
window are handled by the graphics package.  These are interpreted as
{\itshape macros}.  The keystroke values are expanded into magic command-line
commands and executed by the command-line command dispatcher routine.  Two
macros, `{\ttfamily .}' and `{\ttfamily :}' are reserved:  The period
expands to the last command executed from the command line ({\itshape not}
from a macro expansion), and the colon initiates a redirection of keystrokes
from the layout window into the calling terminal.  Magic users quickly
get used to the combination of keystrokes, mouse functions, and
command-line commands.

In the Tcl-based version of Magic, the command-line dispatching is given
over entirely to Tcl, and the dispatching of keystroke events in the layout
window is given over entirely to Tk.  Unfortunately, in relinquishing
these duties, Magic loses some of the effectiveness of its dual-source
input model.  One aspect of this is that Tcl is a line-based interpreter,
and does not recognize anything on the command line until the return key
has been pressed and the entire line is passed to the command dispatcher
routine.  Without any understanding of character-based input, it is
difficult to impossible to directly edit the command line from outside the
calling terminal, because it is the terminal, and not Tcl, which interprets
keystrokes on a character-by-character basis.

The way around this problem is to use a {\itshape console}, which is an
application that runs in the manner usually expected by Tk, in that it is
a GUI-driven application.  The interpreter is split into {\itshape master}
and {\itshape slave} interpreters (see the Tcl documentation on the
{\bfseries interp} command), with the master interpreter running the
console application and the slave interpreter running the application.
The master interpreter then has character-based control over the command
line, and the Magic dual-source input model can be implemented exactly as
originally designed.  The background DRC function can change the command
line cursor from `{\itshape \%}' to `{\itshape ]}' to indicate that the
DRC is in progress, and user input can be redirected from the layout
window to the console with the `{\bfseries :}' keystroke.

In addition to these functions, the console makes use of Tcl's ability
to rename commands to recast the basic Tcl output function
`{\bfseries puts}' in such a way that output to {\ttfamily stdout} and
{\ttfamily stdin} can be handled differently.  In the {\bfseries TkCon}
console, output to {\ttfamily stdout} is printed in blue, while output
to {\ttfamily stderr} is printed in red.  Magic makes use of Tcl's
output procedures so that returned values and information are printed
in blue, while warnings and error messages are printed in red.  The
console also implements command-line history and cut-and-paste methods.
The console command-line history replaces the embedded {\itshape readline}
implementation in magic.

The {\bfseries TkCon} console is a placeholder for what is intended to
be a ``project manager'' console, with functions more appropriate to
the Electronic Design Automation suite of Tcl-based tools.  In general,
the menu functions which are displayed on the TkCon console are not
of particular interest to the Magic user, with the exception of the
{\bfseries History} menu, which can be used to re-enter previously
executed commands, and the {\bfseries Prefs} menu, which includes
highlighting options and a very useful calculator mode.

\section{Tk Graphics Methods}

Because the graphics under Tcl are managed by the Tk package, the only
graphics options which can be compiled are the X11 and the OpenGL
options.  There are numerous differences between these graphics
interfaces as they exist under Tcl and under the non-Tcl-based version.
The two primary differences are the way windows are generated and the
way commands are sent to specific windows.  In Tcl-based magic, a
layout window does not have to be a top-level application window.
by using the command syntax
\starti
   \ii {\bfseries openwindow} {\itshape cellname tk\_pathname}
\endi

An unmapped window can be generated, corresponding to the Tk window
hierarchy specified by {\itshape tk\_pathname} (see the Tk documentation
for the specifics of Tk window path name syntax).  This window is then
mapped and managed by Tk commands.  Using this method, a magic window
can be embedded inside a ``wrapper'' application, an example of which
has been done with the GUI wrapper invoked with the ``{\ttfamily -w}''
command-line argument.  Extensions of magic window commands have been
added so that the wrapper can control the window frame, including the
scrollbars and title.

Whenever a window is created, Magic creates a Tcl/Tk command with the
same name as the window (this is conventional practice with Tk widgets).
Magic and Tcl commands can be passed as arguments to the window
command.  Such commands are executed relative to the specific window.
This applies to all of magic's window-based commands, including
instructions such as {\bfseries move}, {\bfseries load}, and so forth.
Commands which apply to all windows will automatically be sent to all
windows.  These commands are used primarly by the wrapper GUI, but are
also called (transparently to the end user) whenever a command is
executed from a layout window via any macro, including the
`{\bfseries :}' command-line entry.  In addition, however,
they may be called from the command line to perform an action
in a specific window.  By default (in the absence of a wrapper GUI),
magic's windows are named {\itshape .magic1}, {\itshape .magic2}, and
so forth, in order of appearance, and these names are reflected in the
title bar of the window.  So it is equivalent to do
\starti
   \ii {\bfseries :move s 10}
\endi
from layout window `{\itshape .magic2}' (where the colon indicates the
key macro for command-line entry), and 
\starti
   \ii {\bfseries .magic2 move s 10}
\endi
typed in the console.

\section{Tutorial Examples}

Example sessions of running magic in various modes are presented below,
along with examples of methods specific to each mode.  In the examples
below, prompts are shown to indicate the context of each command.
The {\itshape \#} sign indicates the shell prompt, {\itshape (gdb)}
indicates the GNU debugger prompt, and {\itshape \%} indicates the Tcl
(i.e., Magic) prompt.

\medskip
\noindent {\bfseries \itshape Example 1: Standard Magic Execution} \\

Run Tcl-based magic in its most basic form by doing the following:
\starti
   \ii {\itshape \#} {\bfseries magic -noconsole tut2a} \\
\endi

Magic looks generally like its traditional form, except that the
command-line prompt is the Tcl `{\itshape \%}' prompt.  It should be
possible to write commands from either the terminal or using the
colon keystroke, and it is possible to partially type a command
after the colon keystroke and finish the command inside the terminal,
but not vice versa.  Enabling the colon keystroke may require setting
``Allow SendEvents'' mode on the calling terminal.

\medskip
\noindent {\bfseries \itshape Example 2: Console-based Magic Execution} \\

Run the TkCon console-based magic by doing the following:
\starti
   \ii {\itshape \#} {\bfseries magic tut2a} \\
\endi

The layout window will still look like the usual, basic form.  However,
the calling terminal will be suspended (unless the application is
backgrounded;  however, if backgrounding the application, be aware that
any output sent to the terminal will hang the application until it is
foregrounded) and the Tcl prompt will appear in a new window, which is
the {\bf console}.  The console may have a slightly different appearance
depending on the graphics mode used.  For instance, magic has historically
had difficulties running in 8-bit (PseudoColor) graphics mode, because
it installs its own colormap.  Because it does not share the colormap
with the calling terminal, the calling terminal gets repainted in random
colors from magic's colormap when the cursor is in the layout window.  In
unlucky setups, text and background may be unreadable.

In the TkCon console setup, the console is mapped prior to determining
the graphics mode required, so it also does not share the colormap.
However, it is possible to query Magic's colormap to find the location
of specific colors, and repaint the text and background of the console
accordingly.  Thus, the Magic console can be used when the display is
in 8-bit PseudoColor mode, without extreme color remappings in the
console which make it potentially impossible to read the console when
the cursor is in a layout window.

If compiled with both OpenGL and X11 graphics capability, magic will
start in X11 mode by default.  The OpenGL interface can only be
enabled at startup by specifying:
\starti
   \ii {\itshape \#} {\bfseries magic -d OGL tut4x} \\
\endi
The OpenGL interface, in addition to having more solid, vibrant
colors, has an additional feature which draws three-dimensional views
of a layout.  This is discussed in Tutorial~\#??.

\medskip
\noindent {\bfseries \itshape Example 4:  The Magic Wrapper GUI} \\

The magic GUI interface is invoked by starting magic with the {\bfseries
-w} option:
\starti
   \ii {\itshape \#} {\bfseries magic -w tut2b} \\
\endi

The immediately noticeable differences are the layer toolbar on the side,
and the menu and redesigned title bar on the top.  Experimenting with
some mouse clicks, the user will note that the magic coordinates of the
cursor box are displayed on the right-hand side of the titlebar.

The toolbar contains one example of each layer defined in the magic
technology file.  Position a box on the screen, then put the cursor
over a layer icon and press the middle mouse button.  This paints the
layer into the box.  You will also notice that the name of the layer
is printed in the title bar while the cursor is over the layer icon.

The icons have other responses, too.  The first and third mouse buttons
respectively show and hide the layer in the layout.  This works with
labels, subcell boundaries, and error paint (the top three layer icons)
as well as with regular paintable layers.  In addition to mouse button
responses, the buttons invoke various commands in reponse to keystrokes.
Key `p' paints the layer;  key `e' erases it.  Key `s' selects the layer
in the box while key `S' unselects it.

The menubar has three items, {\bfseries File}, {\bfseries Cell}, and
{\bfseries Tech}.  Button {\bfseries File} pops up a menu with options
to read and write layout, read and write CIF or GDS output, open a new
layout window or close the existing one, or flush the current edit cell.
Buttons {\bfseries Cell} and {\bfseries Tech} reveal transient windows
showing information about the cell hierarchy and the technology file,
respectively.  The cell hierarchy view is only available if the Tcl/Tk
package {\bfseries BLT} has been compiled and installed on the system.
If so, it gives a tree view of the cell hierarchy and allows specific
cells to be loaded, edited, and expanded, or viewed to fit the window.
This can be expecially useful for querying the cell hierarchy of GDS
files, which do not declare top-level cells like CIF files do.

The technology manager window reports the current technology file, its
version and description, and the current CIF input and output styles,
the current extraction style, and the current value of lambda in microns.
The technology file and the CIF input and output styles and extraction
style are also buttons which can be used to select a new style or technology
from those currently available.  Clicking on the current extract style, for
instance, gives a list of the styles which have been specified in the
current technology file.  Clicking on one of the entries in the list makes
it the new current extract style.

Cell hierarchy and technology manager windows should be closed by clicking
on the ``Close'' button at the bottom, not by invoking any titlebar functions
(which will probably cause the whole application to exit).

\medskip
\noindent {\bfseries \itshape Example 5: Batch-mode Magic} \\

Unlike previous versions of magic, it is not necessary to have a layout
window present to run magic.  Magic may be invoked in ``batch mode'' by
the following command-line:
\starti
   \ii {\itshape \#} {\bfseries magic -nowindow tut2b} \\
\endi

Note however, that most magic commands expect a window to be present to
execute the function.  The main use for batch mode is for wrapper-type
applications to delay opening a layout window until one is requested.
For example:
\starti
   \ii {\itshape \#} {\bfseries magic -nowindow tut2b} \\
   \ii {\itshape \%} {\bfseries toplevel .myframe} \\
   \ii {\itshape \%} {\bfseries openwindow tut2b .myframe.mylayout} \\
   \ii {\itshape \%} {\bfseries pack .myframe.mylayout} \\
   \ii {\itshape \%} {\bfseries .myframe.mylayout box 0 0 12 12} \\
   \ii {\itshape \%} {\bfseries .myframe.mylayout select area poly} \\
   \ii {\itshape \%} {\bfseries wm withdraw .myframe} \\
   \ii {\itshape \%} {\bfseries wm deiconify .myframe} \\
   \ii {\itshape \%} {\bfseries .myframe.mylayout closewindow} \\
\endi

Note that this is the basic setup of the standard GUI wrapper, albeit with
much greater sophistication.  The standard GUI wrapper is generated
entirely by script, which can be found in the library directory
{\bfseries \$\{CAD\_HOME\}/lib/magic/tcl/wrapper.tcl}.

\medskip
\noindent {\bfseries \itshape Example 6:  Tcl-Magic under the Debugger} \\

When running under Tcl, Magic cannot be debugged in the usual
manner of executing, for instance, ``{\bfseries gdb magic}'',
because the main executable is actually the program ``{\bfseries
wish}''.  To run Magic with debugging capability, it is necessary
to do the steps below.  In the following, it is assumed that
Magic has been installed in the default location {\ttfamily
CAD\_HOME=/usr/local/}, and that the GNU debugger {\ttfamily gdb}
is the debugger of choice.
\starti
   \ih {\itshape \#} {\bfseries gdb wish} \\
   \ih {\itshape (gdb)} {\bfseries run} \\
   \ih {\itshape \%} {\bfseries source /usr/local/lib/magic/tcl/magic.tcl} \\
   \ii . \\
   \ii . \\
   \ii . \\
   \ih {\itshape \%} {\itshape (type Control-C in the terminal window)} \\
   \ih {\itshape (gdb)} {\bfseries break TxTclDispatch} \\
   \ih {\itshape (gdb)} {\bfseries cont} \\
   \ih {\itshape \%} {\bfseries paint m1} \\
   \ih \\
   \ih {\ttfamily Breakpoint 1, TxTclDispath (clientData=0x0, argc=2,}\\
   \ii   {\ttfamily argv=0xbffff400)}\\
   \ih {\ttfamily \ \ \ \ \ at txCommands.c:1146} \\
   \ih {\ttfamily 1146 \ \ \ \ \ \ \ \ DRCBreak();} \\
   \ih {\itshape (gdb)} \\
\endi

Command-line arguments can be passed to magic through the Tcl
variables {\ttfamily argc} and {\ttfamily argv}.  The integer
value {\ttfamily argc} must match the number of entries passed
in the {\ttfamily argv} list.  For example, do the following:
\starti
   \ii {\itshape \#} {\bfseries gdb wish} \\
   \ii {\itshape (gdb)} {\bfseries run} \\
   \ii {\itshape \%} {\bfseries set argc 4} \\
   \ii {\itshape \%} {\bfseries set argv \{-w -d OGL tut1 \}} \\
   \ii {\itshape \%} {\bfseries source /usr/local/lib/magic/tcl/magic.tcl} \\
\endi

\end{document}
